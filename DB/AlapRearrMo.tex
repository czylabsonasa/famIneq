%AlapRearrMo
Foglalkozzunk a jobboldali egyenlőtelenséggel:
$$
\sum_{i=1}^{n} a_{i}b_{\varphi(i)} \le \sum_{i=1}^{n} a_{i}b_{i}
$$
Most $n-1$ lépésben átalakítjuk a $\varphi$ permutációt az identikusra.
Jelentse $(a_{i},b_{j})$ azt hogy az aktuális sorrendben $a_{i}$ és 
$b_{j}$ {\it párok} a összegben. Legyen $k\ge 1$, és tegyük fel, 
hogy már megtettünk $k-1$ lépést, azaz $(a_{i},b_{i})$ 
teljesül $k=\Tolig{1,}{,k-1}$ (ez az első lépés előtt teljesül (üres halmaz)).
$(a_{k},b_{i})$-ben ha $k=i$ akkor továbbmegyünk a következő lépésre.
Egyébként $b_{i}\ge b_{k}$ és $i>k$ a feltevések miatt, és $(a_{j},b_{k})$ valamely $j>i$-re. 
Cseréljük fel $b_{i}$-t és $b_{k}$-t, ekkor az összeg változása:
$$
-a_{k}b_{i}+a_{k}b_{k} - a_{j}b_{k} + a_{j}b_{i}=
(a_{k}-a_{j})(b_{k} - b_{i}) \ge 0,
$$
tehát az aktuális összeget nem csökkentjük, amikor a leírt módon egy új 
elemet a helyére viszünk.
\par A bal oldali egyenlőtlenséghez eljuthatunk a fenti módon, vagy alkalmazzuk 
a jobboldali egyenlőtlenséget a $c_{i}=-b_{i}$ sorozatra.
