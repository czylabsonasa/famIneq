   \begin{align*}
   ab+ac+bc=a^{2}+b^{2}+c^{2}=\\ 
   =\frac{a^{2}+b^{2}}{2}+\frac{a^{2}+c^{2}}{2}+\frac{b^{2}+c^{2}}{2}\ge \\
   \ge |ab|+|ac|+|bc|\ge |ab+ac+bc|
   \end{align*}
   Látjuk, hogy mindenütt egyenlőség kell hogy legyen, ezért 
   \nameref{AlapAMGMFa} miatt $|a|=|b|,|a|=|c|,|b|=|c|$. Ha valamelyik nulla 
   akkor mindegyik az, így feltehetjük hogy egyik sem az. Ha valamelyik kettő előjele 
   különbözik, pl $a=-b$ akkor $ab=-b^2<|ab|=b^2$, holott
   $$
   ab+ac+bc=|ab|+|ac|+|bc|
   $$
   Tehát a számok egyforma előjelűek is.
