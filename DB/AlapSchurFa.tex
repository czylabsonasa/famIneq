   Legyenek $a,b,c\ge 0$ és $r>0$ valós számok. Ekkor
   \begin{align*}
   a^{r}(a-b)+b^{r}(b-a) &\ge 0 \\
   a^{r}(a-b)(a-c)+b^{r}(b-a)(b-c)+c^{r}(c-a)(c-b)  &\ge 0
   \end{align*}
   Egyenlőség az első esetben pont akkor van, ha $a=b$, a másodiknál 
   pont akkor ha $a=b=c$ vagy valamelyik kettő egyenlő a harmadik pedig nulla.
