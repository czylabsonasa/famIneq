   Először homogenizálunk, azaz megpróbáljuk eltüntetni a fokszámkülönbségeket:
   \begin{align*}
   4(a^{3}+b^{3}+c^{3})+24abc \ge 1=(a+b+c)^{3} = \\
   a^{3}+b^{3}+c^{3} + 3a^{2}(b+c)+3b^{2}(a+c)+3c^{2}(a+b)+6abc 
   \end{align*}
   Rendezve:
   \begin{align*}
   a^{3}+b^{3}+c^{3}+6abc \ge \\
   a^{2}(b+c)+b^{2}(a+c)+c^{2}(a+b)
   \end{align*}
   \nameref{AlapSchurFa} $r=1$ miatt valóban fennáll az egyenlőtlenség.
   Mivel egyenlőségnél fennáll a 
   $$
   a^{3}+b^{3}+c^{3}+6abc = a^{3}+b^{3}+c^{3}+3abc = \\
   a^{2}(b+c)+b^{2}(a+c)+c^{2}(a+b)
   $$
   ezért szükséges, hogy $abc=0$ legyen, ehhez elég ha az egyik nulla, 
   amit összevetve a \nameref{AlapSchurFa} egyenlőség feltételével: 
   valamelyik nulla, a másik kettő megegyezik-et kapunk, ami elég is.
